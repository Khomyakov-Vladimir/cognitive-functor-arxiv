\documentclass[12pt, a4paper]{article}
\usepackage[utf8]{inputenc}
\usepackage[T1]{fontenc}
\usepackage[english, russian]{babel}
\usepackage{amsmath, amssymb, amsthm}
\usepackage{graphicx}
\usepackage{tikz-cd}
\usepackage{geometry}
\usepackage{hyperref}
\geometry{margin=1in}
\hypersetup{
    colorlinks=true,
    linkcolor=blue,
    filecolor=magenta,      
    urlcolor=cyan,
}

\title{Cognitive Functor and Topos Collapse: \\ A Variational Model of Ontological Projection}
\author{Владимир Хомяков \and ИИ-ассистент}
\date{\today}

\begin{document}

\maketitle

\begin{abstract}
Мы представляем формальную модель когнитивной проекции как функтора между онтологическими и эпистемическими топосами. Используя вариационный принцип, основанный на ELBO, мы интерпретируем коллапс онтологической структуры в когнитивно сжатый топос как потерю информации, управляемую KL-дивергенцией. Реализация модели использует гиперграфовые нейронные энкодеры и демонстрирует согласованность с теорией предсказывающего кодирования и колмогоровской сложностью. Обсуждаются следствия для интерпретации квантовой механики.
\end{abstract}

\section{Топосная модель когнитивной редукции}

\subsection{Онтологический и когнитивный топосы}

Пусть $\mathcal{ONT}$ -- топос онтологических состояний:
\begin{itemize}
\item Объекты: $\varphi \in \mathcal{H}_{\text{ont}} \subseteq \mathbb{R}^N, N \gg 1$
\item Морфизмы: унитарные операторы $U: \varphi \rightarrow \varphi'$
\end{itemize}

Когнитивный топос $\mathcal{COG}$:
\begin{itemize}
\item Объекты: $x = F(\varphi) \in \mathbb{R}^d, d \ll N$
\item Морфизмы: ограниченные линейные отображения
\end{itemize}

\subsection{Функториальная проекция логики}

\begin{equation}
F: \mathcal{ONT} \rightarrow \mathcal{COG}
\end{equation}

Нарушение закона исключенного третьего:
\begin{equation}
F^*(\neg S) \neq \neg F^*(S)
\end{equation}

\section{Вариационная модель сжатия}

\subsection{Оптимизация на основе ELBO}

\begin{equation}
\text{ELBO} = \mathbb{E}_{q_\theta(x|\varphi)}[\log p(\varphi|x)] - D_{KL}(q_\theta \parallel p)
\end{equation}

\subsection{Информационный коллапс топоса}

Функтор потерь:
\begin{equation}
\mathcal{L}[\phi, F] = \mathrm{KL}(P(\phi) \parallel Q(F(\phi))) + \lambda \cdot \|F(U\phi) - VF(\phi)\|^2
\end{equation}

\section{Реализация и результаты}

\subsection{Архитектура гиперграфового энкодера}

\begin{figure}[h]
\centering
\includegraphics[width=0.8\textwidth]{figures/gnn_architecture.pdf}
\caption{Схема GNN-энкодера для преобразования $\varphi \rightarrow x$}
\end{figure}

\subsection{Визуализация KL-ландшафта}

\begin{figure}[h]
\centering
\includegraphics[width=0.8\textwidth]{figures/kl_landscape.pdf}
\caption{Распределение $D_{KL}$ в пространстве $\mathcal{COG}$ (логарифмическая шкала)}
\end{figure}

\section{Обсуждение}

\subsection{Связь с интерпретациями КМ}

\begin{table}[h]
\centering
\begin{tabular}{|l|c|c|c|}
\hline
 & SF & QBism & RQM \\
\hline
Онтология & Функтор & Субъективность & Отношения \\
\hline
Потери & $F$-редукция & Нет & Частичные \\
\hline
\end{tabular}
\caption{Сравнение интерпретаций}
\end{table}

\subsection{Философские следствия}

Аналогия с принципом Маха:
\begin{equation}
\text{Инерция } \sim F(\text{полная онтология})
\end{equation}

\section*{Благодарности}
Авторы благодарят участников дискуссий за ценные замечания.

\bibliographystyle{unsrt}
\bibliography{references}

\end{document}